% --- Preamble
\documentclass[9pt,aspectratio=169]{beamer}            
% possible pixel values: 8 -- 12 (default), 14, 17, 20pt
% possible aspect ratios: 1610, 149, 54, 43 (default), 32

\usetheme{AnnArbor}
\usecolortheme{beaver}
\usefonttheme{structuresmallcapsserif}
% \useinnertheme{}
% \useoutertheme{}

% --- Packages
\usepackage[normalem]{ulem}             % wrapping underline with UTF8
\usepackage{soul}                       % \ul wraps around new line, no UTF8
\usepackage{mathtools}                  % provides a nicer arrows in math mode
\usepackage{chronology}                 % for simple timelines
\usepackage{qtree}                      % for trees (easier syntax than TikZ)
\usepackage{booktabs}                   % for professional tables
\usepackage{fancybox}                   % fancy boxes!

% --- Title Page Information
\title[short title]{Example Presentation}
\subtitle[short subtitle]{Subtitle}
\author[abbreviated name]{Me}
\institute[short name]{Long Institute Name}
\date{\today}
% \logo{\includegraphics[scale=0.1]{logo}}
% \titlegraphic{\includegraphics[scale=0.1]{logo}}

% --- Add toc to every new section
\AtBeginSection[] {
    \begin{frame}
        \frametitle{Outline}
        \tableofcontents[currentsection]
    \end{frame}
}

\begin{document}

% --- Title Page
\frame{\titlepage}

% --- Table of Contents
\frame{\frametitle{Outline}\tableofcontents[pausesections]}

% --- Slide 1
% include section/subsection for toc only!
\section[short title]{Long Section Title}
\subsection[short title]{Long Subsection Title}
\begin{frame}                       % option "plain" for no decoration
    \frametitle{First Slide}
    \framesubtitle{This is a very simple slide.}
    Whatever content the slide may have\ldots
\end{frame}

% --- Slide 2
\section{Second Section}
\begin{frame}
    \frametitle{Second Slide}
    Whatever content the slide may have\ldots
\end{frame}

% --- Slide 3
\begin{frame}
    \begin{description}
        \item[Item 1] A description of Item 1.
        \item[Item 2] A description of Item 2.
    \end{description}
\end{frame}

% --- Slide 4
\begin{frame}
    \begin{itemize}
        \item One item.
        \item A second item.
    \end{itemize}
\end{frame}

% --- Slide 5
\begin{frame}
    \frametitle{Demonstration of Enumerations.}
    \begin{enumerate}
        \item One item.
            \pause
        \item A second item.
            \pause
        \item A third item.
            \pause
        \item All items \ldots
            \pause
        \item \ldots are shown \ldots
            \pause
        \item \ldots sequentially.
    \end{enumerate}
\end{frame}

% --- Slide 6
\section{Demonstrating blocks.}
\subsection{Simple blocks.}
\begin{frame}
    \frametitle{A simple block.}
    \begin{block}{Block Title.}
        All this text is within the block.
    \end{block}
\end{frame}

% --- Slide 7
\subsection{Alert blocks.}
\begin{frame}
    \frametitle{An alert block.}
    \begin{alertblock}{Block Title.}
        All this text is within the block.
    \end{alertblock}
\end{frame}

% --- Slide 8
\subsection{Example blocks.}
\begin{frame}
    \frametitle{An example block.}
    \begin{exampleblock}{Block Title.}
        Text within the block!
    \end{exampleblock}
\end{frame}

% --- Slide 9
\section{Columns}
\begin{frame}
    \frametitle{This Slide Demonstrates Columns.}

    \begin{columns}
        \begin{column}[]{.5\textwidth}
            \begin{block}{First Title.}
                Whatever the content of this FIRST block is. Wrapping is done in a smart way, though!
            \end{block}
        \end{column}

        \begin{column}[]{.5\textwidth}
            \begin{block}{Second Title.}
                Whatever the content of this SECOND block is. Wrapping is done in a smart way, though!
            \end{block}
        \end{column}
    \end{columns}

    \begin{center}
        The good-old ``center'' still works.
    \end{center}
\end{frame}

% --- Slide 12
\section{Table Demonstration}
\begin{frame}
    \frametitle{A simple, yet beautiful table.}
    \begin{center}
        \begin{tabular}{c|c|c}
            \toprule
            label 1 & label 2 & label 3 \\
            \midrule
            cell 1 & cell 2 & cell 3 \\
            cell 4 & cell 5 & cell 6 \\
            \bottomrule
        \end{tabular}
    \end{center}
\end{frame}

% --- Slide 11
\section{Fancy Boxes.}
\begin{frame}
    \frametitle{Fancy, fancy.}
    \shadowbox{Sample Text.}
    \fbox{Sample Text.}
    \doublebox{Sample Text.}
    \ovalbox{Sample Text.}
    \Ovalbox{Sample Text.}
\end{frame}

% --- Slide 12
\begin{frame}
    Fine.
\end{frame}

\end{document}
